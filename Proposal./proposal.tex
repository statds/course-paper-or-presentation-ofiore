\documentclass[12pt]{article}
\usepackage{filecontents}
\usepackage{natbib}

\title{Proposal}


% Add author information below. Communicating author is indicated by an asterisk, the affiliation is shown by superscripted lower case letter if several affiliations need to be noted.
\author{
Owen Fiore
}

\pagestyle{empty}

\begin{document}

\section{Introduction}

At the 2022 World Track Championships in Eugene Oregon, Devon Allen was disqualified in the
Men’s 110 meter hurdle final after registering a reaction time of 0.099 seconds, 0.001 
seconds faster than what is allowed.  Following the games, bloggers on the running website 
LetsRun scrutinized the results and performed non-statistical analysis of the data \citep{Johnson}
They found that the reaction time data from the 2022 World Track Championships seemed to be faster 
compared to the other datasets they looked at, but did not perform any meaningful statistical 
analysis of the data.  Devon Allen had previously ran the third fastest time in history when 
he ran 12.84 earlier in 2022 and was considered to be one of the favorites to win \citep{Preview}.  His 
disqualification thus prompted questions into whether the timing equipment used at the World
Track Championships was accurate and whether or not Allen actually reacted as fast as the 
timing equipment stated he did.

\section{Specific Aims}
I plan to investigate whether the timing equipment used at the Track and Field World Championships
 produced reaction times different than what times typically are.  Since 1985, Seiko has timed 
 every World Championships with 2022 being no exception\citep{Seiko}.  Thus it would be surprising to discover
 if the timing equipment was producing faster than actual times as the same company was responsible
 for timing the Championships.  By comparing the timing data from 2022 to other World Championship 
 meets, it will be established whether 2022 was an outlier and whether Devon Allen was wrongly 
 disqualified.

\section{Data}

The data set I will be looking at comes from the World Athletics Website which has recorded data
for reaction times in the Men’s 110 meter hurdle since 1999. There are over 800 observations for
reaction times in the heats, semi-finals, and finals for the hurdles since 1999 and there were
67 in 2022.  I intend to look at variables such as year, stage of competition (heats vs finals),
reaction time, total time, and if possible attempt to find environmental data looking at 
temperature, humidity, and elevation.  The data can be found here: \citet{WAData}

\section{Methods}
I plan to use Tukey’s Honestly Significant Difference test to compare pairs of data by year
going back to 1999 to see if any other World Track Championship had significantly different 
results.  Using linear prediction regression in R, I intend to look at whether for each 
championship whether the reaction times got faster in each successive stage of competition.
By comparing data from the heats to the semifinals to the finals, I hope to get a better 
understanding of any confounding variables.  There is also data available for the USA Track
Championships which occurred in June of 2022.  By comparing runners who ran in both the USATF
Championships and the World Championships, a matched pairs t test can be conducted.  Lastly, I
will use a linear regression model that looks at reaction time and the following independent
variables: Year, final time, stage of competition, and any data I can find on the weather at the
meet.

\section{Discussion}
I hope to find data that shows that the 2022 World Championships reaction times were significantly
lower than any other World Championship and that the most significant predictor in determining
reaction time is the year the data was taken.  I have already looked at a very brief sample set
comparing the data from 2019 to 2022 and found that the 2022 reaction times were statistically
significantly faster than in 2019.  If my investigation results in evidence that suggests that
Devon Allen was disqualified due to faulty data, and my work is published I hope it changes the
IAAF ruling of immediately disqualifying all of those who false start, especially under such a
small margin.  If the data is not what I expect, then I do not anticipate much change to occur as
a result.


\bibliographystyle{chicago}
\bibliography{citations.bib}


\end{document}
