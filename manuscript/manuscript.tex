\documentclass[12pt, letterpaper, titlepage]{article}

\usepackage{amsmath}
\usepackage{booktabs}
\usepackage{amsthm}
\usepackage{graphicx}
\usepackage[margin=1in]{geometry}
\usepackage{hyperref}
\hypersetup{colorlinks = true, linkcolor = blue, citecolor=blue, urlcolor = blue}
\usepackage{natbib}
\usepackage{float}
\usepackage{setspace}

\usepackage[pagewise]{lineno}
%\linenumbers*[1]
% %% patches to make lineno work better with amsmath
\newcommand*\patchAmsMathEnvironmentForLineno[1]{%
 \expandafter\let\csname old#1\expandafter\endcsname\csname #1\endcsname
 \expandafter\let\csname oldend#1\expandafter\endcsname\csname end#1\endcsname
 \renewenvironment{#1}%
 {\linenomath\csname old#1\endcsname}%
 {\csname oldend#1\endcsname\endlinenomath}}%
\newcommand*\patchBothAmsMathEnvironmentsForLineno[1]{%
 \patchAmsMathEnvironmentForLineno{#1}%
 \patchAmsMathEnvironmentForLineno{#1*}}%

\AtBeginDocument{%
 \patchBothAmsMathEnvironmentsForLineno{equation}%
 \patchBothAmsMathEnvironmentsForLineno{align}%
 \patchBothAmsMathEnvironmentsForLineno{flalign}%
 \patchBothAmsMathEnvironmentsForLineno{alignat}%
 \patchBothAmsMathEnvironmentsForLineno{gather}%
 \patchBothAmsMathEnvironmentsForLineno{multline}%
}

% control floats
\renewcommand\floatpagefraction{.9}
\renewcommand\topfraction{.9}
\renewcommand\bottomfraction{.9}
\renewcommand\textfraction{.1}
\setcounter{totalnumber}{50}
\setcounter{topnumber}{50}
\setcounter{bottomnumber}{50}

\newcommand{\jy}[1]{\textcolor{blue}{JY: #1}}
\newcommand{\eds}[1]{\textcolor{red}{EDS: (#1)}}


\title{Was Devon Allen Unjustly Disqualified at the 2022 World Track and Field Championships?} 

\author{Owen Fiore\\
%   \href{mailto:owen.fiore@uconn.edu}
% {\nolinkurl{owen.fiore@uconn.edu}}\\
  Elizabeth Schifano\\
  Jun Yan\\[1ex]
  Department of Statistics, University of Connecticut\\
}
\date{}

\begin{document}
\maketitle

\doublespace

\begin{abstract}
  At the 2022 World Track Championships in Eugene Oregon, Devon Allen was disqualified in the
  Men's 110 meter hurdle final after registering a reaction time of 0.099 seconds, 0.001 
  seconds faster than what is allowed.  Following the games, bloggers on the running website 
  LetsRun scrutinized the results and performed non-statistical analysis of the data \citep{Johnson}
  They found that the reaction time data from the 2022 World Track Championships seemed to be faster 
  compared to the other datasets they looked at, but did not perform any meaningful statistical 
  analysis of the data.  Devon Allen had previously ran the third fastest time in history when 
  he ran 12.84 earlier in 2022 and was considered to be one of the favorites to win \citep{Preview}.  His 
  disqualification thus prompted questions into whether the timing equipment used at the World
  Track Championships was accurate and whether or not Allen actually reacted as fast as the 
  timing equipment stated he did.

\bigskip
\noindent\sc{Keywords}:
Devon Allen, Reaction Time, Seiko, 2022 World Track and Field Championships 

\end{abstract}



\section{Introduction}
\label{sec:intro}

Other scholarly papers have discussed track and field reaction times but not in the
context of whether the minimum reaction time barrier is too low or whether or not timing
systems were accurate.  Outside of the blog posts from LetsRun.com, there have not been
any published papers regarding Devon Allen's disqualification in particular.  This is not
neccessarily suprising given that the World Track and Field Championships occurred only four months ago.

The contribution of this paper is to question whether or not Seiko's timing data
was accurate and thus if Devon Allen was disqualified more so because of faulty
equipment rather than reacting illegally fast.  There is little at this point
that could be done to rectify any potential mis-qualification other than to try
to prevent it from happening again.  THis matter needs to be addressed because Devon Allen's disqulification could be a problem that could continue to re-appear.  If athletes
are reacting fast enough that the timing system thinks that they are reacting
illegally fast, then World Athletics needs to address this and possibly change
the minimum reaction time allowed.  Currently the barrier is 0.10 seconds, any time
faster than this would result in a disqualification in the athlete who reacted too
quickly and the race is restarted \citep{Seiko-Timing}.  Any athletes who false start are not given a second chance, they are removed from competition.  This can have massive implications as events like the Outdoor World Track and Field Championship only occur typically once every two years.  It is an opportunity for athletes to earn endorsement deals and prove to sponsors such as Nike and Adidas that
they deserve a contract.  Thus the significance of a disqualification, especially in the finals of a major event such as the World Track and Field Championships, are very
consequential.  






% \begin{figure}[H]
%     \begin{center}
%       \begin{tabular}{||c | c c c | c c c||}[H] 
%        \hline
%        Athlete & USA H & USA S & USA F & World H & World S & World F \\ [0.5ex] 
%        \hline\hline
%        Devon Allen & 0.201 & 0.153 & 0.160 & 0.123 & 0.101 & 0.099 \\ 
%        \hline
%        Trey Cunningham & 0.186 & 0.185 & 0.182 & 0.115 & 0.120 & 0.109 \\
%        \hline
%        Grant Holloway & 0.192 & 0.190 & N/A & 0.147 & 0.128 & 0.124 \\
%        \hline
%        Daniel Roberts & 0.181 & 0.200 & 0.183 & 0.179 & N/A & N/A \\ [0.5ex]
%        \hline
%       \end{tabular}
%       \end{center}
  
% \end{figure}




Additionally it is worth noting that in the semi-finals of the World Track and
Field Championships; Allen's reaction time was 0.101 which is only 0.001 above
the legal limit.  What this suggests is that Allen may have strong reflexes
and be able to react extremely well to the sound of the gun.  Devon Allen is a
very good athlete, as evidenced by his football career at the University of Oregon
and making it onto the Philadelphia Eagles practice squad this past year \citep{Hurley}.
Thus the 0.099 reaction time may not have been a product of Devon Allen
predicting the start of the race but rather a combination of a quick reaction 
and a possibly faulty sensor.

Since 1985 Seiko Holding Corporations has served every World Athletics Championship
as the official timer \citep{Seiko}.  Since 1985, the technology and the ability
to accurately predict measurements: long jump distances, false starts, total time,
reaction time, etc. have all dramatically improved.  Seiko did not start tracking
reaction time as an official measurement until 1999 in the men's and women's 100 meter
hurdles.  Seiko regularly updates their equipment so that they provide cutting edge
technology to the World Track and Field Championships, the highest stakes in the world
of running outside of the Olympics.

%Citation needed in this section
Seiko's technology for detecting reaction times relies on the pressure that athletes
exert on the plate when they push off.  Their systems measure the time differential
between when the "gun" goes off to start the race and when the pressure changes 
\citep{Seiko}  If an athlete has a reaction time under 0.10 seconds, it is deemed a 
false start as it is considered that no athlete can react so quickly \citep{Seiko-Timing}.  
Thus, a reaction time of for example 0.05, suggests that the athlete predicted the gun and it
was luck that caused their abnormally high time.  At the World Championship level,
World Athletics and Seiko want to remove that element of luck and thus impose the 0.1
second barrier.

In 2013, Seiko upgraded their timing equipment for sprinting events (100 meter hurdles
falls under this catergory) \citep{WorldAthletics_2013}.  Since 1999, the highest reaction 
times in the men's 100 meter hurdle were recorded in 2013.  That is not to say that the higher 
times were caused by Seiko's equipment, but the two may be related.  It is worth noting that prior
to 2022, Seiko again upgraded its technology but for its jump management system.
\citep{Pilianidis}



\section{Data}
\label{sec:data}
Data was copied from the World Athletics website and pasted into an excel
spreadsheet. The data covers the men's 110 meter hurdles and the women's 100
meter hurdles from 1999 to 2022.  The variable of prediction interest is reaction
time, and the predicting variables are year, stage of competition (heats, 
semifinal, final), total time, and gender.  The data can be found in the appendix:.  
Factors such as name, country, and identification number were
not considered important for the purposes of this research besides the times for
Devon Allen and the other United States athletes mentioned above.  The variable "Stage"
refers to whether the observation occurred during the "heats" (Preliminary round) which
is denoted by "H" in the data, "S" for "semifinals" (Of which there are either 2 or 3 heats each year), and "F" for finals. The variable "Gender" has levels "M" and "F" which
stand for "male" and "female" respectively.  The variable "Batch" takes a numerical value and every heat of every round from 1999 to 2022 was assigned it's own number.  This
was added because of suspected issues that reaction time may be correlated with other runners reacting quickly (one runner reacting quickly may result in other runners reacting quickly).

One of the most interesting things found was that the reaction times from the USA
athletes in the World Championship 110 meter hurdles, was that they were significantly
faster than at the USA Track Championships.  From June 23-26 2022, USA held it's Track
and Field Championship to decide who to send to the World Championships.  Thus, a baseline
is able to be established for the four USA athletes who competed in at least one round
of both events: Trey Cunningham, Daniel Roberts, Grant Holloway, and Devon Allen. Every
athlete reacted faster in all of the World Track and Field Championship races compared
to the USA Track and Field Championship races. Devon Allen raced three times at
the USA Track and Field Championship and three times at the World Track and Field 
Championship.  His reaction times were significantly faster in the World Track and
Field Championship compared to the USA Track and Field Championship as were his
teamates.  This comparison suggests that the timing system used between the two
meets may have been meaningfully different to the point where Devon Allen was
disqualified more because of faulty equipment rather than because he reacted
too quickly.  This table below highlights the difference in reaction times for
Devon Allen, Trey Cunningham, Grant Holloway, and Daniel Roberts. "USA" denotes
the USA Track and Field Championships, "World" denotes the World Track and Field
Championships, "N/A" was used for any athlete that did not compete and thus
did not register a reaction time.  All numbers listed in the table are the reaction
time for the respective athletes in seconds.

\begin{center}
  \begin{tabular}{||c | c c c | c c c||} 
   \hline
   Athlete & USA H & USA S & USA F & World H & World S & World F \\ [0.5ex] 
   \hline\hline
   Devon Allen & 0.201 & 0.153 & 0.160 & 0.123 & 0.101 & 0.099 \\ 
   \hline
   Trey Cunningham & 0.186 & 0.185 & 0.182 & 0.115 & 0.120 & 0.109 \\
   \hline
   Grant Holloway & 0.192 & 0.190 & N/A & 0.147 & 0.128 & 0.124 \\
   \hline
   Daniel Roberts & 0.181 & 0.200 & 0.183 & 0.179 & N/A & N/A \\ [0.5ex]
   \hline
  \end{tabular}
  \end{center}


\section{Methods}
\label{sec:methods}
The main purpose of this paper is to determine whether or not Devon Allen was
unfairly disqualified because of his reaction time. 

The main statistical model used in the paper is a gamma mixed effects model, which is
in the family of Generalized Linear Mixed Model (GLMM).  A GLMM is a type of model that
supports inputs that are from different distributions.  The model that we use is for the
Gamma distribution for the fixed effects part of the model and a random effects part.  In
the model used in the paper, the fixed effect comes in the form of an intercept, and then
either a year effect, batch effect, or both are used.  Initially a linear mixed effects
model was attempted to be implemented, but due to a right hand skew in the reaction times, a gamma distribution seemed better equipped to model the data (The right
skew is not particularly suprising as it is not possible to register a reaction time below 0.0 seconds, resulting in a lower bound but no upper bound for outliers). In the code, the link parameter under Gamma() was changed to be "log" as logarithmic functions are better equipped to deal with high outliers. 

There are several different data sets that could be used and explored in determining 
whether or not Devon Allen was unjustly disqualified.  One of the best examples of this
is the women's 100 meter hurdles. Although the distance of the race is 10 meters shorter
than the men's race, the reaction time data is still important and can be used to
essentially double the data size.  However, one of the largest issues with this is that on average, men are able to react quicker than women.  Thus the data may end up following
a bimodal distribution which would be difficult to model and not helpful for determining
if Devon Allen's reaction time was the result of a faulty system or if it is an argument
for chaning the 0.10 second minimum barrier that currently exists.  Thus it was decided
to look at men's and women's data separately.  Another decision that was made was to look
at the data from the semifinals and finals together but from the heats or preliminary rounds.  The reason for this was because it was reasoned that in the finals athletes may
be a little more anxious, more "jumpy" and that under stress and they would react faster.
While the preliminary rounds also provide a bulk of the total observations, they are not
as relevant to Allen's disqualification, which occured during the finals.
%%%Include picture 



\section{Conclusion}
\label{sec:conclusion}
%%%Need to discuss how the false start changed from 2007-2009
The Tukey HSD performed on the finals data showed that across the many t tests it
performed, 8 were signifcantly different of which 6 came from 2022 with the other
two came from 2013 which was a high outlier on reaction time.  No other year had
as low as a reaction time as 2022 for the finals, and the individual t test
comparing 2019 and 2022 produced a p value of $2.654\cdot10^{-8}$.  This is an
extermely significant p value and shows that there was a significant difference
betweent the 2019 times and 2022 times, but there does not appear to be a reason
for why the times should be so much higher.  Shoe technology may have improved
during that time, but that would likely be reflected in the total time and not
in the reaction time.  Additionally, other environmental factors such as wind,
heat, etc. also should not impact how quick athletes are to react to the sound
of the gun.  When the analysis of variance was run to compare the linear mixed
effects model and the random effects model, the resulting p value was $3.243\cdot10^{-5}$.
This is a highly significant p value at all alpha levels and it means that there
does appear to be a year effect when predicting reaction time.  From the R output,
the estimate of the intercept was 0.148883, the variance of the intercept was
$8.643\cdot10^{-5}$, the variance of the residuals was $3.719\cdot10^{-4}$.
However, this does not provide sufficient evidence of how strong the year effect
is.  In order to calculate that, the pnorm() function was used, which returned
a p value of 0.0112, which is low but not significant at all alpha levels.  This
means that there is a very great chance that the year effect is significant but
that we expect that if we repeatly sampled, the chance of observing a test
statistic higher would be only 0.0112.  

Devon Allen's disqulification at the World Track and Field Championships was
possibly the result of both a faulty timing equipment and Devon Allen
reacting extremely quickly to the start.  The code and data from R showed that
the 2022 World Track and Field Championships were a low outlier compared to
every other year in terms of average reaction time.  Addtionally, it was shown
through a linear mixed effects model that there is a year effect that is significant
in determining reaction time.  However, this practically does not make much sense
as there is no reason for so much random variation without much of a trend.
There has not been a consistent decrease or increase since 1999, much of the data
for reaction times has been random and unpredictable.  Thus while it seems easy
to conclude that Devon Allen was wrongly disqualified, that may not neccessarily
be the case.


\section{Appendix}
\label{sec:appendix}
%Put downloadable data spreadsheet
Here is World Athletics website with the data: \url{https://www.worldathletics.org/results/world-athletics-championships}.
Here is the data for the USA Track and Field Championships: \url{https://www.flashresults.com/2022_Meets/Outdoor/06-23_USATF/}


\bibliographystyle{chicago}
\bibliography{citations.bib}


\end{document}
